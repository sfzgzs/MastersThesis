
\section{Preliminary Lemmas}
The following theorem lets us use a simpler characterization for effective locally uniform continuity of a function
that does not depend on any effective open exhaustion.
Leveraging this characterization,
we can reduce the problem of proving effective local uniform continuity w.r.t an open exhaustion for its domain, to a simpler problem.
This characterization is especially used later in Theorems \ref{thm::effective_local_uniform_continuity_addition} and \ref{thm::effective_local_uniform_continuity_multiplication}
for proving effective local uniform continuity of the addition and multiplication of functions.
% \edcomm{WK}{(Just curious: Is this still different from effective non-local uniform continuity?)}

\begin{definition}[Local continuity witness]
\label{def::local-continuity-witness}
    Let $f:\Reals\to\Reals$. A recursive function $N:\Rationals\times\Rationals\times\Nats\to\Nats$ is called a \emph{local continuity witness for $f$}\index{local continuity witness} iff for any $a,b\in\Rationals$ with $[a,b]\subseteq\dom(f)$ and $k\in\Nats$, we have
    \[ 
        \forall x,y \in (a,b) \quad \abs{x-y}<2^{-N(a,b,k)} \then \abs{f(x)-f(y)}<2^{-k}.
    \] 
\end{definition}

\begin{theorem}[Alternative characterization of effective local uniform continuity]
    \label{thm::alternative_characterization_for_eluc}
    Let $f:\Reals\to\Reals$ have an open domain with an open exhaustion $(U_1, U_2, \ldots)$. Then
    $f$ is effectively locally uniformly continuous with respect to $(U_1, U_2, \ldots)$ if and only if there is a local continuity witness for $f$.
\end{theorem}
\begin{proof}
    \begin{itemize}
        \item[($\Rightarrow$)]
            Assuming $f$ is effectively locally uniformly continuous with respect to an open exhaustion $(U_1, U_2, \ldots)$ for $\dom(f)$, we get a recursive function $M:\Nats\times\Nats\to\Nats$ such that for all $k,i\in \Nats$ and all $x,y \in U_i$, 
            \[
            \abs{x-y}<2^{-M(k,i)} \then \abs{f(x) - f(y)}< 2^{-k}.
            \]
            We need to prove the existense of a local continuity witness for $f$. To do this, we present an algorithm computing a recursive function $N:\Rationals\times\Rationals\times\Nats\to\Nats$ that takes $a,b\in \Rationals$ and $k\in \Nats$ with $[a,b]\subseteq \dom(f)$ as inputs and returns a natural number such that 
            \[
                \abs{x-y}<2^{-N(a,b,k)} \then \abs{f(x) - f(y)}< 2^{-k}.
            \]
            Since $(U_i)_{i \in \Nats}$ covers $\dom(f)$, there is a stage in which $[a,b]$ is covered. By definition of an effective open exhaustion, there is a recursive map which delivers the sequence of stages of $(U_i)_{i \in \Nats}$. Hence we can enumerate all stages until we find a stage $s$ which contains $[a,b]$, and then we can output $N(a,b,k) = M(k,s)$. This guarantees that 
            \[
            \forall x,y \in (a,b) \quad \abs{x-y}<2^{-N(a,b,k)} \then \abs{f(x)-f(y)} < 2^{-k}.
            \]
        \item[($\Leftarrow$)]
            Let us assume $f$ has a local continuity witness $N:\Rationals\times\Rationals\times\Nats\to\Nats$.
            We need to define a recursive function $M:\Nats\times\Nats\to\Nats$ such that 
            \[
                \forall x,y\in U_i \quad \abs{x-y}<2^{-M(k,i)} \then \abs{f(x)-f(y)} < 2^{-k}.
            \]
            Since we have an open exhaustion for $\dom(f)$, we have a recursive function listing the intervals in each stage. Let $(a_1, b_1, \ldots, a_{n_s}, b_{n_s})$ be the endpoints of the components of stage $s$.
            Let us enumerate the gaps between intervals as $g_1, g_{n_s-1}$ with $g_j\in\Rationals$ for each $j\in \{1, \ldots, n_s-1\}$. 
            Then, we define 
            \[
            M(k,s) \stackrel{def}= \max (\curlybrac{N(a_m, b_m, k) \mid 1<m<n_s}\cup \{ \lceil 1/g_j \rceil \mid 1\leq j \leq n_s-1 \}).
            \]
            Consider $x,y \in  U_s$. Then if $\abs{x-y}<2^{-M(k,s)}$, this means that $x$ and $y$ must be on the same interval in the open exhaustion. We also have that $\abs{x-y}<2^{-N(a_m, b_m, k)}$ for $1<m<n_s$, this guarantees that $\abs{f(x)-f(y)} < 2^{-k}$,
            and hence, proves that $f$ is locally uniformly continuous with respect to $(U_i)_{i \in \Nats}$.
    \end{itemize}
\end{proof}
                
\noindent
We take this theorem as justification for being able to talk about just
``effective local uniform continuity'',
instead of having to talk about ``effective local uniform continuity w.r.t an open exhaustion''.

Theorem \ref{thm::alternative_characterization_for_eluc} gives us an alternative characterization of acceptable functions, i.e.:
\begin{corollary}
    A function $f:\Reals\to\Reals$ is \emph{acceptable} iff:
    \begin{enumerate}[(i)]
        \item The domain of $f$ is the union of an effective open exhaustion, and
        \item The function $f$ has a local continuity witness. \index{local continuity witness}
    \end{enumerate}
\end{corollary}
\begin{lemma}
\label{lemma::WCC-approximation-gives-rec-func--for-cont-on-monotone-intervals}
% \edcomm{FG}{There is a problem with possibly too much handwaiving here (also in \ref{thm::OEX_by_WCC_approximability}). I am technically (mis?)using the \WhileCC-procedure in this theorem as a function of type $\Rationals\times \Nats \to \Rationals$ to give me an approximation for $f(x)$ om $x\in\Rationals\cap\dom(f)$... can I get away with this?}
% \edcomm{WK}{I think, yes: You are implicitly using that recursive interpreter of WhileCC that I mentioned.}
% \edcomm{FG}{Do I need to formally define the recursive interpreter then?}
% \edcomm{WK}{I currently don't think you'd nbeed that.}
    Let $F:\real\times\nat\to\real$ be a \WhileCC-procedure approximating the function $f:\Reals\to\Reals$, with $f$ monotone on its domain, then $f$ is effectively locally uniformly continuous.
\end{lemma}
\begin{proof}
    By Theorem \ref{thm::alternative_characterization_for_eluc}, it suffices to give a recursive function $N:\Rationals\times\Rationals\times\Nats\to\Nats$ that for any $a,b,\in \Rationals$ with $[a,b]\subseteq \dom(f)$ satisfies 
    \[
        \forall x,y \in (a,b) \quad \abs{x-y}<2^{-N(a,b,k)} \then \abs{f(x)<f(y)}<2^{-k}.
    \]
    We present an algorithm to compute $N(a,b,k)$ for any $a,b\in\Rationals$ with $[a,b]\subseteq \dom(f)$.
    We begin by giving an informal description of the algorithm:
    \begin{enumerate}
        \item 
            Start with a counter $c = 0$.
        \item
            Divide $[a,b]$ into segments $[a_1, a_2],[a_2, a_3], \cdots, [a_{c_n-1}, a_{c_n}]$ of length at most $2^{-c}$.
        \item
            Check whether for all $i\in\{1,\ldots, c_n-1\}$,
            there exists some $q_1 \in F^{\algebraR}(a_i, c)$
            and $q_2\in F^{\algebraR}(a_{i+1}, c)$ where
            \[
                \abs{q_2 - q_1}+2^{-c-1}<2^{-k}.
            \]
            If this is not satisfied, increase $c$ by one, and go to step 2. Otherwise, return $c$.
    \end{enumerate}
    The algorithm we just defined computes the following functions:
    \begin{align*}
        \textrm{intervals}(a,b,i,c) & =
            \begin{cases*}
                [a+2^{-c}i, a+2^{-c}(i+1)] & if $2^{-c}(i+1) < b$\\
                [a+2^{-c}i, b] & otherwise\\
            \end{cases*}\\
        N(a,b,k) & = \min_{c\in\Nats}\ \forall i \in \{0,\ldots, \lceil(b-a)/2^{-c}\rceil \} \\ 
        & \qquad\exists q_1\in F^{\algebraR}(\textrm{intervals}(a,b,i,c), c)\\
        & \qquad \exists q_2\in F^{\algebraR}(\textrm{intervals}(a,b,i+1,c), c)\\
        & \qquad \abs{q_2 - q_1}+2^{-c-1}< 2^{-k}.
    \end{align*}
    Intuitively, $\textrm{intervals}(a, b, i, c)$ divides the interval $[a, b]$ into a finite set of sub-intervals of maximum length $2^{-c}$, returning the $i$th such interval. Note that $\abs{q_2 - q_1}+2^{-c-1}$ is an overestimation of how much the value of $f$ changes within the interval $[a_i, a_{i+1}]$. Since the function is monotone over $I$, the maximum change in any interval of length at least $2^{-c}$ is less than the estimation, and hence step 3 will guarantee that 
    \[
     \forall x,y \in (a,b) \quad \abs{x-y}<2^{-N(a,b,k)} \then \abs{f(x)-f(y)}<2^{-k}.
    \]
\end{proof}