
\begin{theorem} 
\label{thm::OEX_logarithm}
   The function $\ln(x)$ is \effectivelyOpen{}.
   % \edcomm{WK}{This is one of the times where I'd like to have an index. Where is ``function $\_$ is effectively open'' defined? And, separate question, what is it needed for?}\edcomm{FG}{Added a bit of explanation at the beginning of Section \ref{sec4::subsection::basics}. Do you know of a way I could make an index without having to $\backslash$index every single time I use the important words?}
   % \edcomm{WK}{You'd index only the important occurrences, in particular the defining occurrence.}
\end{theorem}
\begin{proof}
    Let us take any arbitrary open set $U$ with an effective open exhaustion.
    We want to come up with an effective open exhaustion for $\ln^{-1}(U) = \exp(U)$.
    
    Using Remark \ref{remark::OEX_for_interval_implies_OEX_for_all_open_sets}, we only need to prove that the preimage of any open interval $I =(a,b)$ with $a,b \in \mathbb{Q}$ has an effective open exhaustion.
    
    The pre-image function for $\ln$ is $\exp$.
    
    We know that
    \begin{itemize}
        \item The function $\exp$ is monotonically increasing.
        \item The function $\exp$ is \WhileCC-approximable (Using Theorem \ref{theorem::exp}).
        \item The function $\exp$ is defined on any arbitrary interval $(a,b)$.
    \end{itemize}
    Hence, Theorem \ref{thm::OEX_by_WCC_approximability} gives us an open exhaustion for $\exp(I)$, and this completes the proof.
\end{proof}
