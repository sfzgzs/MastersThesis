
\section{Computability on \texorpdfstring{$\mathbf{\Reals}$}{R}: \texorpdfstring{$\mathbf{\alpha}$}{alpha}-tracking Computability}
\label{sec::alpha-computability}
In this section, we review the concept of $\alpha$-computability, which is central to the definition of elementary functions defined in \ref{sec::elementary-functions}, and an equivalence lemma by \citet{ModelOfCompForPartFunc_MingQuanFuAndJeffZucker}.

\begin{definition}[Standard enumeration of $\Rationals$, \citep{ModelOfCompForPartFunc_MingQuanFuAndJeffZucker}]
\label{def::standard-enumeration}
    Let $\alpha:\Nats\to\Rationals$. We call $\alpha$ a \textit{\standardenumeration{}} for $\Rationals$ if $\alpha$ is bijective, and the field operations on $\Rationals$ ($+, \cdot, -, /$) are recursive under $\alpha$, i.e.,
    \begin{itemize}
        \item there is a recursive function $\mathit{add}:\Nats\times\Nats\to\Nats$ such that 
        \[
        \forall n_1, n_2\in \Nats\quad \alpha(n_1)+\alpha(n_2) = \alpha(\mathit{add}(n_1, n_2)),
        \]
        \item there is a recursive function $\mathit{mult}:\Nats\times\Nats\to\Nats$ such that 
        \[
        \forall n_1, n_2\in \Nats\quad \alpha(n_1)\cdot\alpha(n_2) = \alpha(\mathit{mult}(n_1, n_2)),
        \]
        \item there is a recursive function $\mathit{sub}:\Nats\times\Nats\to\Nats$ such that 
        \[
        \forall n_1, n_2\in \Nats\quad \alpha(n_1)-\alpha(n_2) = \alpha(\mathit{sub}(n_1, n_2)),\]
        and
        \item there is a recursive function $\mathit{div}:\Nats\times\Nats\to\Nats$ such that 
        \[
        \forall n_1, n_2\in \Nats\quad n_2\neq 0 \then  \alpha(n_1)/\alpha(n_2) = \alpha(\mathit{div}(n_1, n_2))
        \]
    \end{itemize}
\end{definition}
\begin{theorem}
    There is a \standardenumeration{} for $\Rationals$.
\end{theorem}
\begin{proof}
Let us consider G\"odel's pairing function $g(x/y)=2^x(2y+1)-1$. This is a bijection between positive rationals and $\Nats$. We can tweak this to give us a bijection $f$ between $\Rationals$ and $\Nats$. Let us define
\begin{equation*}
    f(x/y) =
    \begin{cases*}
        2(g(x/y)+1)-1 & if $x/y>0$ \\
        2g(x/y) & if $x/y<0$, 
    \end{cases*}
\end{equation*}
This maps positive rationals to odd natural numbers and negative rationals to evens.
The bijection $f^{-1}$ is a standard enumeration for $\Rationals$ since
the functions $\mathit{add}, \mathit{mult}, \mathit{sub}, \mathit{div}$ are easily proven to be recursive.
\end{proof}
\begin{definition}[Computable reals codes $\Omega$, \citep{ModelOfCompForPartFunc_MingQuanFuAndJeffZucker}]
    Let $\langle -,-\rangle:\Nats\times\Nats\to\Nats$ be a recursive encoding of pairs.
    The set $\Omega \subset \Nats$ is the set of all codes $\langle e, m \rangle$ such that
    \begin{itemize}
        \item \( e \) is an index for a total recursive function \( \phi_e : \mathbb{N} \to \mathbb{N} \) generating a Cauchy sequence\index{$\phi$}
        \[
        \alpha(\phi_e(0)), \alpha(\phi_e(1)), \alpha(\phi_e(2)), \ldots
        \]
        of elements of \( \Rationals \).
        \item \( m \) is the index of a computable modulus of convergence \( \phi_m : \Nats\totalTo\Nats \), ensuring:
        \[
        \forall k, l \geq \phi_m(n), \quad |\alpha(\phi_e(k)) - \alpha(\phi_e(l))| < 2^{-n}.
        \]
    \end{itemize}
\end{definition}
\begin{definition}[$\alpha$-(tracking) computable reals $\Reals_c$, \citep{ModelOfCompForPartFunc_MingQuanFuAndJeffZucker}]
    For any \standardenumeration{} $\alpha$, we define an enumeration of computable reals $\bar{\alpha}:\Omega\to\Reals$ to be \index{$\phi$} 
     \[
        \bar{\alpha}( \langle e, m \rangle ) = \lim_{i\rightarrow \infty} \alpha(\phi_e(i)).
    \]
    The range of $\bar{\alpha}$ is called \textit{the set of $\alpha$-tracking computable reals (or \alphacomputablereals{})} and is denoted with $\Reals_c$. 
\end{definition}

\begin{definition}[$\alpha$-tracking function, \citep{ModelOfCompForPartFunc_MingQuanFuAndJeffZucker}]
    Let \( \alpha \) be a \standardenumeration{} of \( \Rationals \). For functions \( f : \mathbb{R} \to \mathbb{R} \) and \( \tau : \mathbb{N} \to \mathbb{N} \), \( \tau \) is an \( \alpha \)-tracking function for \( f \) if:
    \begin{enumerate}
        \item If \( f(\bar{\alpha}(k)) \) is defined, then \( \tau(k) \) is defined, and:
        \[
        f(\bar{\alpha}(k)) = \bar{\alpha}(\tau(k)).
        \]
        \item If \( f(\bar{\alpha}(k)) \) is undefined, then \( \tau(k) \) is undefined.
    \end{enumerate}
\end{definition}

\begin{definition}[$\alpha$-(tracking) computability, \citep{ModelOfCompForPartFunc_MingQuanFuAndJeffZucker}]
\label{def::alpha-bar-comp}
    The function $f:\mathbb{R}\to\mathbb{R}$ is an \emph{$\alpha$-tracking computable (also called an \alphacomputablefunction{})} iff it has a recursive $\alpha$-tracking function. 
\end{definition}


\begin{corollary}
    \label{corollary::soundness}
    \footnote{This is a corollary to Theorem (A) in \citet{AbstractVSConcreteComputationOnMetricPartialAlgebras_TuckerZucker_2004}
    }
    Let $f:\Reals\to\Reals$ be a \WhileCC-approximable function and $\alpha:\Nats\to\Rationals$ be any standard enumeration for $\Rationals$.
    Then $f$ is $\alpha$-computable.
\end{corollary}
Note that corollary \ref{corollary::soundness} implies that
for any two standard enumerations $\alpha_1$ and $\alpha_2$ for $\Rationals$,
the set of $\alpha_1$-computable and $\alpha_2$-computable functions coincide under the assumption of \WhileCC-approximability.


Using the Definition \ref{def::alpha-bar-comp}, we can easily see the $\alpha$-computability of the constant function(for $\alpha$-computable reals) and the identity function: 
\begin{lemma}
\label{lemma::const_is_alpha_comp}
    The constant function $f(x) = c$ for any \alphacomputablereal{} $c$ is $\alpha$-computable.
\end{lemma}
\begin{proof}
    Using the definition of ${\alpha}$-computability, it suffices to show that the constant function has a recursive $\alpha$-tracking function.\\
    Since $c$ is an $\alpha$-computable real, this means $c \in \mathbb{R}_c$. 
    ‌Since by definition, $\alpha$ is a surjection, $\alpha^{-1}(c) \neq \emptyset$. So let's take an arbitrary $y \in \alpha^{-1}(c)$ and let us define $\tau(x) = y$ for all $x$. Then $\tau$ is a recursive $\alpha$-tracking function for $f$.
\end{proof}
\begin{lemma}
\label{lemma::id_is_alpha_comp}
    The identity function $id(x) = x$ is ${\alpha}$-computable.
\end{lemma}
\begin{proof}
    Using the definition of ${\alpha}$-computability, it suffices to show that the identity function has a recursive $\alpha$-tracking function.\\    
    let $\tau$ be the identity function on $\mathbb{N}$. Then $\tau$ is a recursive $\alpha$-tracking function‌ for $id(x)$.
\end{proof}
