\subsection{\WhileCC-Approximability of sin(x)}
    Since the function $\sin(x)$ is analytic on $\Reals$, we can compute sufficiently many terms of its Taylor series to get a ``sufficiently close'' approximation.
    Considering the Taylor expansion of $\sin(x)$ expanded around $x = 0$, we get:
    \begin{equation}
    \label{eq::sin}
        \sin(x) = \sum_{n=0}^{\infty} (-1)^n \frac{x^{2n+1}}{(2n+1)!}
    \end{equation}
    
    To approximate $\sin(x)$, we need to figure out how many iterations of the summation above need to be calculated to achieve the desired precision.
    Let us define the \WhileCC-procedure $\isProperIndex : \nat\times\real\times\nat\to\bool$ below:
   \[
   \isProperIndex \equiv 
        \BLOCK{
        \PROC\\
            \INDENT{
            \IN N:\nat\ x:\real\ c:\nat\\
            \OUT r:\bool}\\
        \BEGIN\\
            \INDENT{
            r := 
            \BLOCK{
            {x^{N}}\slash{\factorial(N)} \realLess\ {2}^{-c}\\
            \land\ 2 \times x < N \\
            \land\ 2 \times - x < N}\\
            \RETURN r}\\
        \END} 
   \]
   The semantics of the procedure $\isProperIndex$ is as below:
    \[
    \isProperIndex^\algebraR(N,x,c) = \curlybrac{ \true } \quad \iff \quad 
    x^{N}/N! < 2^{-c}\ \land\ 2\abs{x} < N.
    \]
    The procedure $\isProperIndex$ returns true only if calculating the first $N$ terms of the series will produce a sufficiently precise approximation.
    
   \begin{lemma}
   \label{lemma::isProperIndex_sin}
       Let $c\in \Nats$ and $x \in\Reals$. Then for any $N\in\Nats$
       \[
       \isProperIndex^\algebraR(N, x, c) = \{\true\} \then \abs{\sin(x) - \sum_{n=0}^N (-1)^n \frac{x^{2n + 1}}{(2n + 1)!}}<2^{-c}.
       \]
   \end{lemma}
   \begin{proof}
    Let us assume $\isProperIndex^\algebraR(N, x, c) =\{\true\}$. By definition, this means that 
    \begin{enumerate}[(i)]
        \item 
        \label{item::ispropindex::1-sin}
            $x^{N}/N! < 2^{-c}$, and
        \item 
        \label{item::ispropindex::2-sin}
            $2\abs{x}< N$.
    \end{enumerate}
    Then,
           \begin{align*}
                \abs{\sin(x) - \sum_{n = 0}^{N}{(-1)^n\frac{x^{2n + 1}}{(2n + 1)!}}}
                & =\abs{\sum_{n = N + 1}^{\infty}(-1)^n\frac{x^{2n + 1}}{(2n + 1)!}}\\
                & =\abs{\sum_{k = 1}^{\infty}(-1)^{n + k}\frac{x^{2N+2k+1}}{(2N+2k+1)!}}\\
                & \leq\sum_{k = 1}^{\infty}\frac{\abs{x}^{2N+2k+1}}{(2N+2k+1)!} \tag{by triangle inequality} \\
                & < \sum_{k = 1}^{\infty}\frac{1}{2^{N+2k+1}}\cdot \frac{\abs{x}^N}{N!}\tag{by assumption (\ref{item::ispropindex::2-sin}) and Lemma \ref{lemma::sum::g}}\\
                 & < \sum_{k = 1}^{\infty}\frac{1}{2^k}\cdot\frac{\abs{x}^{N}}{N!}\\
                 & =1 \cdot \frac{\abs{x}^{N}}{N!} \\
                 &< 2^{-c} \tag{by assumption (\ref{item::ispropindex::1-sin})}
            \end{align*}
    \end{proof}
    \begin{lemma}
    \label{lemma::existence::sin}
        For any $c\in \Nats$ and arbitrary $x\in\Reals$, there is some $N\in \Nats$ for which
        \[\quad \isProperIndex^\algebraR(N, x, c) = \{ \true\}.\]
    \end{lemma}
   \begin{proof}
         By Theorem \ref{thm::spivak_theorem}, choosing $\epsilon = 2^{-c}$, there exists an index $N>2\abs{x}$ sufficiently large such that $x^{N}/N! < 2^{-c}$.
   \end{proof}
   Now we know that for arbitrary inputs $x\in\Reals$, we \textit{can} find the necessary number of terms to calculate to get the desired precision. 
    
    \begin{definition}
    \label{def::sin}
    The \WhileCC-procedure
    \[ 
    \Sine\equiv 
            \BLOCK{
            \PROC \\
                \INDENT{
                \IN x:\real\ c:\nat \\
                \AUX \counter:\nat\ N:\nat\ \summ:\real}\\
            \BEGIN \\
                \INDENT{
                \counter := \natZero \\
                \summ := \realZero \\
                N := \CHOOSE (k : \nat) : \isProperIndex(k,x,c)\\
                \WHILE \counter \natLess N \DO \\
                    \INDENT{
                    \summ := 
                        \BLOCK{\summ\ +\\
                            ({-1}^{\counter} \times {x^{2 \times \counter + 1}}\slash{\factorial(2 \times \counter + 1)})}\\
                    \counter := \counter + 1}\\
                od \\
                \RETURN \summ}\\
            \END}\\
    \]
    of type $\real\times\nat\to\real$ has the semantics
    \[
        \Sine^\algebraR(x,c) = \mleft\{ \sum_{n=0}^N \frac{x^{2n+1}}{(2n+1)!}  \mid \isProperIndex(N,x,c) =\{ \true\} \mright\}.
    \] 
\end{definition}
The procedure $\Sine$ approximates $\sin(x)$ for any $x\in\Reals$ by calculating the first $N$ terms of the series \ref{eq::sin} where $N$ is chosen to satisfy $\isProperIndex$.
\begin{corollary}
\label{cor::sin}
    For any $c\in\Nats$ and $x\in\Reals$, 
    Using Lemma \ref{lemma::existence::sin} and Definition \ref{def::sin}, it immediately follows that 
    \[ \Sine^\algebraR(x,c) \neq \emptyset.
    \]
\end{corollary}

\begin{lemma}
\label{lemma::sin-neighbourhood}
    For any $n\in\Nats$ and $x\in\Reals$ we have 
    \[
    \Sine^\algebraR(x,n) \subseteq \Nbd(\sin(x), 2^{-n})
    \]
\end{lemma}
\begin{proof}
    Follows immediately from Definition \ref{def::sin} and Lemma \ref{lemma::isProperIndex_sin}.
\end{proof}