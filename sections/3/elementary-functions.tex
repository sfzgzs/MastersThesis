\section{Elementary Functions}
\label{sec::elementary-functions}
Elementary functions appear in different computational contexts. 
\begin{definition}[Elementary functions]
\label{def::elementary_functions}
    The \elementaryfunctions{} on $\Reals$ \cite[page 17]{OrdinaryDifferentialEquations_Morris_Tenenbaum_1985} are partial functions defined by expressions built up from 
    \begin{itemize}
        \item \alphacomputablereals{} (see Remark \ref{why-alpha-computable} and Section \ref{sec::alpha-computability}), and
        \item the variable $x$
    \end{itemize}
    and, by applying (repeatedly) the basic operations below on elementary functions $f,g$:
    \begin{enumerate}[(i)]
        \item
        addition %\footnote{$\simeq$ used from here on
        % \edcomm{WK}{Are previous uses, e.g., in $\mathcal{R}$, something different?}
        % is the Kleene equality, meaning either both sides of the equality are defined and are equal, or both sides are undefined. \edcomm{WK}{\ldots{} and you are using some funny symbol to indicate undefinedness\ldots}} 
        (i.e. $(f+g)(x) = f(x) + g(x)$) 
        \item
        multiplication (i.e. $(f\cdot g)(x) = f(x)g(x)$)
        \item
        division  (i.e. $\division_f(x) = \frac{1}{f(x)}$ where $\frac{1}{0}$ is undefined) 
        \item
        exponential (i.e. $\exp_f(x) = e^{f(x)}$)
        \item
        logarithm (i.e. $\ln_f(x) = \ln(f(x))$
        \item
        $\sin_f(x) = \sin(f(x))$
        \item
        $n$-th roots: root\(_{n,f}(x) = \sqrt[n]{f(x)}\) where $0< n \in \Nats$
        \item
        $\arcsin_f(x) = \arcsin(f(x))$
    \end{enumerate}
\end{definition}
In this thesis, we make the following modifications to the natural definition of some of the functions above,
to work with a computable part of $\Reals$, and also to make sure the domains of those functions are open:
\begin{itemize}
    \item We define $\sqrt[n]{x} = 0$ for $x < 0$ when $n$ is even.
    \item We extend the definition of $\arcsin(x)$ to be $\frac{\pi}{2}$ for $x > 1$ and to be $-\frac{\pi}{2}$ for $x<-1$.
\end{itemize}
From here on, the term \textit{elementary functions} will be used to refer to the modified version of unary elementary functions.

\begin{remark}
\label{why-alpha-computable}
    Note that the original definition of elementary functions by \cite[page 17]{OrdinaryDifferentialEquations_Morris_Tenenbaum_1985}
    involves expressions built up from arbitrary (possibly non-computable) constants.
    We cannot approximate non-computable constants,
    so we choose to omit them from our definition of \elementaryfunctions{}
    and only work with \alphacomputablereals{}.
\end{remark}
