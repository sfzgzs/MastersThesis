\newcommand\mVal{\mathit{mVal}}
\section{Division}
In this section, we construct a \WhileCC-procedure that approximates $\division_f(x)=1/f(x)$ using a \WhileCC-procedure that approximates the functions $f$.
\label{sec4::subsection::division}
\begin{definition}
  Let $F:\real\times\nat\to\real$ be a \WhileCC-procedure approximating the function $f:\Reals\to\Reals$, and and $x\in \Reals$. Then
the \WhileCC-procedure 
    \[
    \mleft({1}/ F(-,-)\mright) \equiv
                \BLOCK{
                \PROC \\
                    \INDENT{
                    \IN x:\real\ n:\nat  \\
                    \AUX \res:\real\ \chosenVal:\nat\ m:\real}\\
                \BEGIN \\
                    \INDENT{
                    \chosenVal := \CHOOSE (k : \nat) : 
                    \BLOCK{
                    \PROC\\
                        \INDENT{
                        \IN k:\nat\ x:\real\\
                        \AUX \res:\bool}\\
                    \BEGIN\\
                        \INDENT{
                        \IF k \natEq 1 \THEN\\
                            \INDENT{
                            \res := 0< F(x)}\\
                        \ELSE\\
                            \INDENT{
                            \res := F(x) < 0}\\
                        \FI\\
                        \RETURN \res}\\
                    \END}\\
                \IF \chosenVal \natEq 1 \THEN\\
                    \INDENT{
                        \res, m := \CHOOSE 
                        \BLOCK{
                        (q:\real,\mVal:\real) :\\  
                        \BLOCK{
                        \mathsf{abs}(q - \frac{1}{\mVal}) < 2^{-n} \\
                        \INDENT{
                        \boolAnd}\\
                        (0 < \mVal<F(x))} \\
                        }\\
                        }\\
                \ELSE\\
                    \INDENT{
                        \res, m := \CHOOSE 
                        \BLOCK{
                        (q:\real,\mVal:\real) :\\
                        \mathsf{abs}(q - \frac{1}{\mVal}) < 2^{-n} \\
                        \INDENT{
                        \boolAnd}\\
                        (F(x)<\mVal < 0)}}\\
                \FI\\
                \RETURN res}\\
                \END}\\
    \]

of type $\real\times\nat\to\real$ has the semantics 
\[
 \begin{aligned}[t]
    \mleft({1}/{F(x, n)}\mright)^\algebraR = \{ & q\in \Rationals \mid \exists m\in \Rationals \quad 0<mid<f(x) 
        & \land 0<\abs{q - ({1}/{m})}<2^{-n} \}\\ & \cup {} \\
        \{ &q\in \Rationals \mid \exists m\in \Rationals \quad f(x)<m<0 & \land 0<\abs{q - ({1}/{m})}<2^{-n}\}.
 \end{aligned}
\]
Note that instead of $({1}/{F(-,-)})^\algebraR(x,n)$ we write $({1}/{F(x,n)})^\algebraR$.
\end{definition}
\noindent
We handle the case of $0<f(x)$ and $f(x)<0$ separately. In each case we choose a rational value $q$ that is sufficiently close to $1/f(x)$. Since we cannot calculate the exact value of $f(x)$, we choose an approximation of it called $m$ here and choose $q$ to be sufficiently close to $m$.
\begin{lemma}
\label{lemma::condition-division}
    Let $f:\Reals\to\Reals$. Then for any $q,m \in\Rationals$ we have 
    \begin{enumerate}[(i)]
        \item 
            \label{div-cond-1}
            $0<m<f(x) 
            \land 0<\abs{q - ({1}/{m})}<2^{-n} \then \abs{q - (1/f(x))}<2^{-n}$
            , and 
        \item
            \label{div-cond-2}
            $f(x)<m<0  \land 0<\abs{q - ({1}/{m})}<2^{-n}\then \abs{q - (1/f(x))}<2^{-n}$.
    \end{enumerate}
\end{lemma}
\begin{proof}
    First let's assume $f(x)>0$, then
    \begin{align*}
        & 0<\abs{q - {1}/m}<2^{-n} \\
        \then & \abs{q - {1}/{f(x)}}<2^{-n}\tag{since $0<mid<f(x)$}
    \end{align*}
    Now the case where $f(x)<0$ is similar:
    we have
    \begin{align*}
        & 0<\abs{q - {1}/m}<2^{-n} \\
        \then & \abs{q - {1}/{f(x)}}<2^{-n}\tag{since $f(x)<mid<0$}
    \end{align*}
\end{proof}
Now, we go on to prove that the \WhileCC-procedure $(1/F(-,-))$ approximates the function $\division_f(x) = 1/f(x)$.
\begin{claim}
\label{claim::division::1}
    Let $F:\real\times\nat\to\real$ be a \WhileCC-procedure approximating the function $f:\Reals\to\Reals$. Then for any real $x \in \dom(\division_f)$ and for any $n\in \Nats$, we have $ ({1}/{F(x,n)})^\algebraR\neq \emptyset$.
\end{claim}
\begin{proof}
    Since $x \in \dom(\division_f)$. This means that $x\in \dom(f)$ and $f(x)\neq 0$. This means there is always a rational number $m$ between $0$ and $f(x)$. Now no matter what value the chosen non-zero $mid$ has, we can choose a rational number $q$ sufficiently close to ${1}/{m}$ satisfying $\abs{q - {1}/{mid}}<2^{-n}$. Hence $({1}/{F(x,n)})^\algebraR\neq \emptyset$.
\end{proof}
\begin{claim}
\label{claim::division::2}
    Let $F:\real\times\nat\to\real$ be a \WhileCC-procedure approximating the function $f:\Reals\to\Reals$. Then for any real $x \notin \dom(\division_f)$ and for any $n\in \Nats$, we have $ ({1}/{F(x,n)})^\algebraR =\emptyset$.
\end{claim}
\begin{proof}
    Assume $x\notin \dom(\division_f)$. This means that at least $x\notin \dom(f)$ or $f(x) = 0$. 
    If $x\notin \dom(f)$, then $(F(x)>0)^\algebraR = (F(x)<q)^\algebraR = \emptyset$, and hence $({1}/{F(x,n)})^\algebraR = \emptyset$.
    Now let's assume $x\in \dom(f)$ and $f(x) = 0$.
    This means no $0<mid<0$ can be found and hence $({1}/{F(x,n)})^\algebraR = \emptyset$.
\end{proof}
\begin{claim}
\label{claim::division::3}
    Let $F:\real\times\nat\to\real$ be a \WhileCC-procedure approximating the function $f:\Reals\to\Reals$. Then $\forall x\in \dom(f) \textrm{ with } f(x)\neq 0,\ \forall n \in \Nats,\ \forall q\in ({1}/{F(x,n)})^\algebraR :$
    \[
    \abs{{1}/{f(x)} - q} < 2^{-n}.
    \]
\end{claim}
\begin{proof}
Using the definition of the procedure $(1/F(-,-))$, Lemma \ref{lemma::condition-division} immediately implies that $(1/F(x,n))^\algebraR\subseteq \mathbf{Nbd}(1/f(x), 2^{-n})$.
\end{proof}
\begin{theorem}
\label{theorem::division}
    Let $F:\real\times\nat\to\real$ be a \WhileCC-procedure approximating the function $f:\Reals\to\Reals$. Then the \WhileCC-procedure  $({1}/{F(-,-)})$ approximates div$_f$.
\end{theorem}
\begin{proof}
    Follows directly from Lemmas \ref{claim::division::1}, \ref{claim::division::2}, \ref{claim::division::3}.
\end{proof}
