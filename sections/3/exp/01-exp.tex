\subsection{\WhileCC-Approximability of exp(x)}
Since the function $e^x$ is analytic on $\Reals$, we can compute sufficiently many terms of its Taylor series to get a `sufficiently close' approximation. The Taylor expansion of $e^x$ is as follows:
\begin{equation}
\label{eq::exp}
    e^x = \sum_{n=0}^\infty \frac{x^n}{n!}
\end{equation}
\noindent
To approximate $e^x$, we need to figure out how many iterations of the summation above need to be calculated to achieve the desired precision.
Let us define the \WhileCC-procedure $\mathsf{isProperIndex}:\nat\times\real\times\nat\to\real$ below:
\[
    \mathsf{isProperIndex} \equiv  \\
                \BLOCK{
                \PROC  \\
                    \INDENT{
                    \IN N:\nat\ x:\real\ c:\nat \\
                    \AUX \res:\bool}\\
                \BEGIN \\
                    \INDENT{
                    \res := 
                    \BLOCK{
                    x^{N}\slash{\factorial(N)}  < (2^{-c}) \\
                    \boolAnd
                    (2\ \times\ x) < N \\ \boolAnd
                    (2\ \times\ - x) < N}\\
                    \RETURN \res}\\
                \END}
\]
The semantics of the procedure $\mathsf{isProperIndex}$ is as below:
\[
    \mathsf{isProperIndex}^\algebraR(N,x,c) = \curlybrac{ \true } \quad \iff \quad 
    x^n/N! < 2^{-c}\ \land\ 2\abs{x} < N.
\]
The procedure \isProperIndex returns true only if calculating the first $n$ terms of the series will
produce a sufficiently precise approximation.

\begin{lemma}
\label{lemma::isProperIndex_exp}
    For all $N \in \Nats$, $x \in \Reals$, and $c \in \Nats$, we have
    \[
    \isProperIndex^\algebraR(N, x, c) =\{\true\}\then \abs{e^x - \sum_{n = 0}^{N}{\frac{x^n}{n!}}}<2^{-c}
    \]
\end{lemma}
\begin{proof}
    Let us assume $\isProperIndex^\algebraR(N, x, c) =\{\true\}$. By definition, this means that 
    \begin{enumerate}[(i)]
        \item 
        \label{item::ispropindex::1}
            $x^n/N! < 2^{-c}$, and
        \item 
        \label{item::ispropindex::2}
            $2\abs{x}< N$
    \end{enumerate}
    We compute an upper bound for the error of the Taylor series approximation up to the $N$th term:
    \begin{align*}
        \abs{e^x - \sum_{n = 0}^{N}{\frac{x^n}{n!}}} & =  \abs{\sum_{n = N+1}^{\infty}{\frac{x^n}{n!}}}\\
        & =\abs{\sum_{k = 1}^{\infty}\frac{x^{N+k}}{(N+k)!}} \\
        & \leq \sum_{k = 1}^{\infty}\frac{\abs{x}^{N+k}}{(N+k)!} \tag{by triangle inequality}\\
        & < \sum_{k = 1}^{\infty}\frac{1}{2^k}\cdot\frac{\abs{x}^{N}}{N!} \tag{by assumption (\ref{item::ispropindex::2}) and Lemma \ref{lemma::sum::g}}\\
        & =1 \cdot \frac{\abs{x}^{N}}{N!} \\
        & < 2^{-c} \tag{by assumption (\ref{item::ispropindex::1})}
    \end{align*}
\end{proof}
Now we need to make sure that the premise in Lemma \ref{lemma::isProperIndex_exp} can actually be satisfied.

\begin{lemma}
    \label{lemma::existence::exp}
    For any $c\in \Nats$ and arbitrary $x\in \Reals$, there is some $N\in \Nats$ for which
    \[\quad \isProperIndex^\algebraR(N, x, c) = \{ \true\}.\]
\end{lemma}
\begin{proof}
     By Theorem \ref{thm::spivak_theorem}, choosing $\epsilon = 2^{-c}$, there exists an index $N>2\abs{x}$ such that $\frac{x^N}{N!} < 2^{-c}$.
\end{proof}
Now we know that for arbitrary inputs $x$, we can find the necessary
number of terms we must calculate to get the desired precision.
\begin{definition}
\label{expOfX}
    The \WhileCC-procedure
    \[
        \Exp \equiv  \\
                    \BLOCK{
                    \PROC \\
                        \INDENT{
                        \IN x:\real\ c:\nat  \\
                        \AUX \counter:\nat\ N:\nat\ \summ:\real}\\
                    \BEGIN \\
                        \INDENT{
                        \counter := \natZero \\
                        \summ := \realZero \\
                        N := \CHOOSE (k : \nat) : 
                                \isProperIndex(k,x,c) \\
                        \WHILE \counter \natLess N \DO \\
                            \INDENT{
                            \summ := \summ + x^{\counter}\slash \factorial(\counter) \\
                            \counter := \counter + 1}\\
                        \OD \\
                        \RETURN \summ}\\
                    \END}\\
    \]
    of type $\real\times\nat\to\real$ has the semantics
    \[
    \Exp^\algebraR(x,c) = \mleft\{
    \sum_{n=0}^N \frac{x^n}{n!} \mid \isProperIndex(N,x,c)
    \mright\} 
    \]
\end{definition}
    The procedure $\Exp$ approximates $e^x$ for $x\in\Reals$ by calculating the first $N$ terms of the series (\ref{eq::exp}) where $N$ is chosen to satisfy $\isProperIndex$.
\begin{claim}
\label{claim::exponential::1}
    For any real $x \in \Reals$ and for any $n\in \Nats$, we have $\Exp^\algebraR(x, n)\neq \emptyset$.
\end{claim}
\begin{proof}
    Follows directly from the definition of $\Exp^\algebraR$ and Lemma \ref{lemma::existence::exp}.
\end{proof}
\begin{claim}
\label{claim::exponential::2}
    For any $x\in \dom(\exp) = \Reals$, $n\in\Nats$, and $y \in \Exp^\algebraR(x,n)$ we have
    \[
    \abs{e^x - y} < 2^{-n}.
    \]
\end{claim}
\begin{proof}
Follows directly from the definition of $\Exp^\algebraR$ and Lemma \ref{lemma::isProperIndex_exp}.
\end{proof}
\begin{theorem}
\label{theorem::exponential}
    The \WhileCC-procedure $\Exp$ given in Definition \ref{expOfX} approximates $\exp$.
\end{theorem}
\begin{proof}
    Follows directly from Lemmas \ref{claim::exponential::1}, \ref{claim::exponential::2} and the fact that $\dom(\exp) = \Reals$.
\end{proof}